\documentclass[a4paper, 10pt]{article}

\usepackage[utf8x]{inputenc}
\usepackage[english]{babel}
\usepackage{fancyhdr}
\usepackage{hyperref}
\usepackage{url}
\usepackage{graphicx}
\usepackage{ku-forside}
\usepackage{xcolor, colortbl}
\usepackage{cite}
\usepackage{url, hyperref}
\usepackage{amsmath}

%\setlength\parskip{1em}
%\setlength{\parindent}{0pt}
\def\arraystretch{2}

\fancyhead[LO,RE]{XMP\\ Philip Munksgaard}
\fancyhead[LE,RO]{\today}
\pagestyle{fancy}

\titel{Extreme Multiprogramming}
\undertitel{Fall 2012}
\opgave{Final Exam: Programming Report}
\forfatter{\shortstack[l]{P. Munksgaard (240789)}}
\dato{\today}
\vejleder{A. Filinski, B. Vinter}

\newcommand{\conc}{\, \| \,}
\newcommand{\how}[1]{\text{[#1]}}

\begin{document}
\maketitle

\newpage

\section{Implementation}

I have implemented the braking system using Python and PyCSP. 

There are two main processes that make up the braking system: the
\verb+motor+ and the \verb+control+. Both of them have an argument \verb+num+ that
is simply used to identify different processes easily from each other.

The \verb+motor+ process models the motor and the wheel. It keeps track of
\verb+last_speed+, which is the last correct reading of the wheels
turning speed. Say the wheel is moving at a speed of $10$ and suddenly
the motor is told to accelerate to $100$, the motor applies so much
force that the wheel starts spinning. In our model, the wheel will
then immediately start spinning at a speed of $100$, but the last know
``good speed'', \verb+last_speed+, won't change. Consequently, when
the control process figures out that the wheel is spinning and
communicates a new \verb+target_speed+, the motor can decide if the
reduced speed will result in a wheel that is spinning or not. 

\verb+motor+ runs a loop that waits for a \verb+target_speed+ from
\verb+control+ and then decides how to change \verb+last_speed+ as
well as what to report back to \verb+control+. This means, that the
speed only changes once for each \verb+target_speed+ that has been
received, so to go from a speed of $10$ to a speed of $15$, the
\verb+control+ process has to send $5$ \verb+target_speed+s of $15$
and receive $5$ \verb+observed_speed+s (which is what \verb+control+
calls the values passed through it's input channel from the
\verb+motor+). Whether a \verb+wheel+ blocks/spins or not, is decided
by the variable \verb+acc_threshold+.

Furthermore, the \verb+motor+s can receive a \verb+skew+ which can be
used to force desynchronization, and uses the value \verb+epsilon+ to
make python deal correctly with floats (used when skewing).

Then \verb+control+ process contains a loop that receives a
\verb+target_speed+ from a central (in my code it is simply the
main process sending and receiving messages in the test functions) and
calls the \verb+go_to_speed+ function. It also handles the
\verb+NewSpeedException+, which is raised when a new
\verb+target_speed+ is received from central (because of the recursive
nature of \verb+go_to_speed+ it can't be handled in there).

\verb+go_to_speed+ is a function that takes channels from a
central, to and from the \verb+motor+ and another
\verb+control+, an \verb+observed_speed+ and a \verb+target_speed+. It
consists of a big loop that continues until \verb+observed_speed+ is
equal to \verb+target_speed+. First it tries to send the
\verb+target_speed+ to the \verb+motor+ and then receive the updated
observed speed \verb+new_obs_speed+. If the function decides that the
wheel is spinning or blocking, it calls itself recursively with a new
\verb+target_speed+ that is half the difference between the
\verb+target_speed+ and the \verb+observed_speed+.

In case the wheel isn't spinning or blocking, it exchanges
\verb+observed_speed+s with the complementing \verb+motor+. If the
process then decides that the two \verb+motor+s are out of sync, it
decides whether to wait for the other process or continue accelerating.

At all points where \verb+go_to_speed+ can output to the \verb+motor+
it also has the option of receiving a new target speed from the
\verb+control+, allowing new target speeds to be given at any time.

\subsection{Shutdown}

All processes are ended by retiring the channels to and from the
\verb+control+ processes in the central process. Once the control
process tries to access these channels, they in turn will retire the
channels to and from the \verb+motor+ processes thus closing all
remaining processes.

\section{Assesment}

The five functions \verb+test0+, \verb+test1+, \verb+test2+,
\verb+test3+, \verb+test4+ are used to test different scenarios for
the trainbrake system. 

\verb+test0+ is a simple test that spawns ten wheelsets and sends them
all a target speed of 15. There should be no spinning or blocking, nor
any desynchronizations with this test, and the processes should quite
quickly report that they've reached the target speed. Thus
\verb+test0+ tests the basic functionality of the system: setting and
achieving a target speed.

\verb+test1+ tests that the \verb+control+ process can receive
subsequent orders.

\verb+test2+ tests that blocking and spinning works as expected.

\verb+test3+ tests that interrupting the \verb+control+ and giving it
a new target speed works as expected.

\verb+test4+ forces desynchronization, and tests that \verb+control+s
wait for each other.

All tests work as expected, and we can conclude that the processes, at
least for these cases, work as expected.

\end{document}
